\documentclass[xcolor=dvipsnames]{beamer} % en min\'uscula!
\usefonttheme{professionalfonts} % fuentes de LaTeX
\usetheme{Warsaw} % tema escogido en este ejemplo
\usecolortheme[named=blue]{structure}
%%%% packages y comandos personales %%%%
\usepackage[utf8]{inputenc}
% \usepackage{latexsym,amsmath,amssymb} % S\'imbolos
% \usepackage{xcolor}
% \usepackage{MnSymbol}

\usepackage{bussproofs} 

\newcommand{\cf}[1]{\textcolor{blue}{#1}}
\newcommand{\ct}[1]{\textcolor{blue}{#1}}
\newcommand{\cc}[1]{\textcolor{ForestGreen}{#1}}
\newcommand{\ck}[1]{\textcolor{orange}{#1}}

\newcommand{\N}{\ct{\mathbb{N}}}
\newcommand{\ra}{\rightarrow}
\newcommand{\T}{ \ \ \ \ }

 \newcommand{\tjud}[2]
  {\ensuremath{#1\,:\,#2}}
 \newcommand{\depFun}[3] {
  {\ensuremath{\Pi_{(\tjud{#1}{#2})}\,#3\,#1}}
 }
 \newcommand{\cprod}[2]
  {\ensuremath{#1 \times #2}
 }
 \newcommand{\depPair}[3]
 {\ensuremath{\Sigma_{(\tjud{#1}{#2})}\,#3\,#1}
 }

\begin{document}
\beamertemplatenavigationsymbolsempty
\title{Agda, un lenguaje con tipos dependientes}
\author{{Alejandro Gadea, Emmanuel Gunther}\\
\vspace*{0.5cm}}
\date{14 de Noviembre de 2014}
\frame{\titlepage}

\section{Introducción}

\begin{frame}

\begin{block}{ }
 En el desarrollo de software los programas se escriben esperando que
 satisfaga una especificación.
\end{block}

\pause

\begin{exampleblock}{Especificación de head}
 La función \cf{head} toma una lista de elementos no vacía y retorna el primero de ellos.
\end{exampleblock}

\pause

\begin{block}{}
 En la práctica en general no se asegura que un programa cumpla con la especificación.
\end{block}


\end{frame}

\begin{frame}

\begin{block}{ }
 Los sistemas de tipos agregan expresividad a los programas, permitiendo expresar propiedades que deben satisfacer.
\end{block}

\pause

\begin{exampleblock}{head en python}
  \ck{def} \cf{head} (l):
  
  \T \ck{return} l[0]
\end{exampleblock}

\pause

\begin{block}{}
  Podríamos tener en un programa la línea \cf{head} (\cc{true}) (lo cual no cumple una precondición) 
  y tendríamos un error en tiempo de ejecución.
\end{block}

\end{frame}

\begin{frame}
  \begin{block}{}
   En un lenguaje con sistema de tipos fuerte podemos especificar que la función \cf{head} 
   toma como argumento una lista y retorna un elemento:
  \end{block}

  \pause
  
  \begin{exampleblock}{head en haskell}
    \cf{head} :: [a] $\rightarrow$ a
    
    \cf{head} (x : xs) = x
  \end{exampleblock}  

  \pause
  
  \begin{block}{}
    Sin embargo podríamos hacer \cf{head} $[]$, lo cual violaría la precondición de que la lista no sea vacía.
     y tendríamos un error en tiempo de ejecución.
  \end{block}
  

\end{frame}

\begin{frame}
  \begin{block}{}
    Fortaleciendo el sistema de tipos podemos agregar información sobre lo que se espera de un programa:
  \end{block}

  \pause
  
  \begin{exampleblock}{head en agda}
    \cf{head} : \{a : Set\}  $\rightarrow$ (xs : List a) $\rightarrow$ 0 $<$ length xs $\rightarrow$ a \\
    \cf{head} $[]$ ()\\
    \cf{head} (x $ :: $ xs) \_ = x
  \end{exampleblock}  

  \pause
  
  \begin{block}{}
    La función \cf{head} ahora espera una lista de elementos y una \textit{prueba} de que tiene más de un elemento.
  \end{block}

  \pause
  
  \begin{block}{}
    Nunca podría llamarse \cf{head} con una lista vacía ya que no se podría \textit{construir} 0 $<$ length $[]$.
  \end{block}

\end{frame}

\section{Tipos de datos y proposiciones lógicas}

\begin{frame}
  \begin{block}{}
    A los tipos de datos que tenemos en lenguajes como Haskell, se los puede fortalecer teniendo tipos que \textbf{dependen}
    de valores.
  \end{block}
  
  \pause
  
  \begin{block}{Familia de tipos}
    Si $A$ es un tipo de dato, podemos tener una función $B$ que toma un elemento de $A$ y retorna un tipo.
    A una función así se la denomina \textbf{familia de tipos} indexada en $A$.
  \end{block}  

  \pause
    
  \begin{block}{}
    Con este sistema de tipos más fuerte, se puede ver una correspondencia entre las proposiciones lógicas
    y los tipos de datos.
  \end{block}

\end{frame}

\begin{frame}

   \begin{block}{Reglas para el $\wedge$}
      \AxiomC{$p$}
      \AxiomC{$q$}
      \LeftLabel{Introducción de $\wedge$:}
      \BinaryInfC{$p \wedge q$}
      \DisplayProof
      \quad
      \AxiomC{$p \wedge q$}
      \LeftLabel{Eliminación de $\wedge$:}
      \UnaryInfC{$p$}
      \DisplayProof
      \quad
      \AxiomC{$p \wedge q$}
      \UnaryInfC{$q$}
      \DisplayProof
    \end{block}
    
    \pause
    
    \begin{block}{Reglas del tipo producto (tuplas en Haskell)}
      \AxiomC{$\tjud{a}{A}$}
      \AxiomC{$\tjud{b}{B}$}
      \BinaryInfC{$\tjud{(a,b)}{\cprod{A}{B}}$}
      \DisplayProof
      \quad
      \AxiomC{$\tjud{t}{\cprod{A}{B}}$}
      \UnaryInfC{$\tjud{pr_1\,t}{A}$}
      \DisplayProof
      \quad
      \AxiomC{$\tjud{t}{\cprod{A}{B}}$}
      \UnaryInfC{$\tjud{pr_2\,t}{B}$}
      \DisplayProof
    \end{block}


\end{frame}

\begin{frame}

 \begin{block}{}
  Podemos hacer la misma correspondencia con el resto de los conectivos y otros tipos de datos:
 \end{block}

 \pause
 
 \begin{block}{Lógica proposicional y teoría de tipos}
  \begin{itemize}
  \item $\mathbf{True}$ se corresponde con $\mathbf{\top}$ (tipo con exactamente \textbf{un} constructor).
  \item $\mathbf{False}$ se corresponde con $\mathbf{\bot}$ (tipo sin constructores).
  \item $\mathbf{A \wedge B}$ se corresponde con $\cprod{A}{B}$ (tipo de 2-uplas).
  \item $\mathbf{A \vee B}$ se corresponde con $A + B$ (tipo de la unión disjunta).
  \item $\mathbf{A \Rightarrow B}$ se corresponde con $A \rightarrow B$ (funciones de $A$ en $B$).
  \item $\mathbf{A \Leftrightarrow B}$ se corresponde con $\cprod{(A \rightarrow B)}{(B \rightarrow A)}$.
  \item $\mathbf{\neg A}$ se corresponde con $A \rightarrow \mathbf{\bot}$.
\end{itemize}
 \end{block}
 
\end{frame}

\begin{frame}
     \begin{block}{}
      Si pensamos un tipo como una proposición lógica, para probar su validez debemos \textbf{construir un elemento}
      del tipo.
     \end{block}
     
     \pause
     
     \begin{block}{}
      El tipo $\top$ tiene un solo constructor ($\tjud{tt}{\top}$). Por lo tanto \textbf{siempre podemos
      construir} un elemento de $\top$.
      
      Análogamente, $\mathbf{True}$ es \textbf{siempre válido}.
     \end{block}

     \pause
     
      \begin{block}{}
      El tipo $\bot$ no tiene constuctores. Por lo tanto \textbf{nunca podemos construir} un elemento
      de $\bot$
      
      Análogamente, $\mathbf{False}$ \textbf{nunca puede probarse}.
     \end{block}

\end{frame}

\begin{frame}
 \begin{block}{}
  Con la posibilidad de definir familias de tipos podemos definir \textbf{funciones dependientes} y \textbf{pares dependientes}.
 \end{block}
 
 \pause
 
 \begin{block}{}
  Si $A$ es un tipo y $B$ una familia indexada en $A$, tenemos el tipo de funciones dependientes:
  $(\tjud{x}{A}) \rightarrow B\,x$.
 \end{block}
 
 \pause
 
 \begin{block}{}
  Si $A$ es un tipo y $B$ una familia indexada en $A$, tenemos el tipo de pares dependientes:
  $\depPair{x}{A}{B}$. Los elementos de este tipo son pares donde el tipo del segundo elemento \textbf{depende}
  del valor del primero.
 \end{block} 

 \pause
 
\end{frame}


\begin{frame}
 
   \begin{block}{}
  Con estos tipos más complejos, podemos tener una correspondencia con los cuantificadores del cálculo
  de predicados, completando la tabla anterior:
 \end{block}

 \pause

  \begin{block}{Lógica de predicados y teoría de tipos}
   \begin{itemize}
    \item ($\forall x$, $P\,x$) se corresponde con $(\tjud{x}{A} \rightarrow P\,x$.
    \item ($\exists x$, $P\,x$) se corresponde con $\depPair{x}{A}{P}$.
    \end{itemize}
  \end{block}

  \pause
  
  \begin{block}{}
   Gracias a esta correspondencia podemos tener un \textbf{lenguaje de programación} con un sistema
   de tipos que \textbf{permita expresar proposiciones lógicas}.
  \end{block}

\end{frame}


\section{Introducción a Agda}
 
\begin{frame}

\begin{block}{}
Utilizando ejemplos presentemos las diferentes características que
mencionamos antes sobre los sistema de tipos dependientes, utilizando
Agda y cómo éstas nos permite expresar propiedades y diferentes tipos.
\end{block}

\pause

\begin{block}{}
\begin{itemize}

\item Entender por qué \textit{funciona} el \cf{head} presentado anteriormente.\\
\item Presentar una implementación más adecuada utilizando vectores.\\
\item Implementar algunas cosas interesantes aprovechando la definición de los
vectores.
\end{itemize}
\end{block}

\end{frame}

\begin{frame}

\begin{block}{}
Empecemos introduciendo las listas y los naturales.
\end{block}

\pause

\begin{block}{Listas de A}

\ck{data} \ct{List} (A : \ct{Set}) : \ct{Set} \ck{where}\\
\T \cc{$[]$} : \ct{List} A\\
\T \cc{$\_::\_$} : A $\ra$ \ct{List} A $\ra$ \ct{List} A

\end{block}

\pause

\begin{block}{Naturales}

\ck{data} $\N$ : \ct{Set} \ck{where}\\
\T \cc{zero} : $\N$\\
\T \cc{suc}  : $\N$ $\ra$ $\N$

\end{block}

\end{frame}

\begin{frame}
\begin{block}{}
Recordando el tipo del \cf{head}:\\

\

\{A : \ct{Set}\} $\ra$ 
(xs : \ct{List} A) $\ra$ \cc{zero} $<$ \cf{length} xs $\ra$ A\\

\

Nos falta definir el tipo $<$. Es interesante mencionar que para
este tipo podemos empezar a pensar a los valores que se pueden construir
como si fueran pruebas en el sentido matemático.
\end{block}

\pause

\begin{block}{Orden entre naturales}

\ck{data} \ct{\_$\leq$\_} : $\N$ $\ra$ $\N$ $\ra$ \ct{Set} \ck{where}\\
\ \ \ \ \cc{z$\leq$n} : \{n : $\N$\}                 $\ra$ \cc{zero}  \ct{$\leq$} n\\
\ \ \ \ \cc{s$\leq$s} : \{m n : $\N$\} $\ra$ (m$\leq$n : m \ct{$\leq$} n) $\ra$ \cc{suc} m \ct{$\leq$} \cc{suc} n

\end{block}

\pause

\begin{block}{Orden estricto entre naturales}
\cf{$\_<\_$} : $\N$ $\ra$ $\N$ $\ra$ \ct{Set}\\
m \cf{$<$} n = \cc{suc} m \cf{$\leq$} n
\end{block}

\end{frame}

\begin{frame}

\begin{block}{Orden entre naturales}

\ck{data} \ct{\_$\leq$\_} : $\N$ $\ra$ $\N$ $\ra$ \ct{Set} \ck{where}\\
\ \ \ \ \cc{z$\leq$n} : \{n : $\N$\}                 $\ra$ \cc{zero}  \ct{$\leq$} n\\
\ \ \ \ \cc{s$\leq$s} : \{m n : $\N$\} $\ra$ (m$\leq$n : m \ct{$\leq$} n) $\ra$ \cc{suc} m \ct{$\leq$} \cc{suc} n

\end{block}

\begin{block}{Orden estricto entre naturales}
\cf{$\_<\_$} : $\N$ $\ra$ $\N$ $\ra$ \ct{Set}\\
m \cf{$<$} n = \cc{suc} m \cf{$\leq$} n
\end{block}

\begin{block}{head en Agda}
    \cf{head} : \{A : \ct{Set}\}  $\ra$ (xs : \ct{List} A) $\ra$ 
    \cc{zero} $<$ \ct{length} xs $\rightarrow$ A \\
    \pause
    \cf{head} \cc{$[]$} p$_{[]}$ = ¿?\\
    \cf{head} (x \cc{$::$} xs) p$_{::}$ = ¿?
  \end{block}  

\end{frame}

\begin{frame}

\begin{block}{Orden entre naturales}

\ck{data} \ct{\_$\leq$\_} : $\N$ $\ra$ $\N$ $\ra$ \ct{Set} \ck{where}\\
\ \ \ \ \cc{z$\leq$n} : \{n : $\N$\}                 $\ra$ \cc{zero}  \ct{$\leq$} n\\
\ \ \ \ \cc{s$\leq$s} : \{m n : $\N$\} $\ra$ (m$\leq$n : m \ct{$\leq$} n) $\ra$ \cc{suc} m \ct{$\leq$} \cc{suc} n

\end{block}

\begin{block}{Orden estricto entre naturales}
\cf{$\_<\_$} : $\N$ $\ra$ $\N$ $\ra$ \ct{Set}\\
m \cf{$<$} n = \cc{suc} m \cf{$\leq$} n
\end{block}

\begin{block}{head en Agda}
    \cf{head} : \{A : \ct{Set}\}  $\ra$ (xs : \ct{List} A) $\ra$ 
    \cc{zero} $<$ \ct{length} xs $\rightarrow$ A \\
    \cf{head} \cc{$[]$} p$_{[]}$ = ¿?\\
    \cf{head} (x \cc{$::$} xs) p$_{::}$ = x
  \end{block}  

\pause

\begin{block}{}
Podemos preguntarnos que valores (o pruebas) podrían ser p$_{[]}$ y p$_{::}$
\end{block}

\end{frame}

\begin{frame}

\begin{block}{Orden entre naturales}

\ck{data} \ct{\_$\leq$\_} : $\N$ $\ra$ $\N$ $\ra$ \ct{Set} \ck{where}\\
\ \ \ \ \cc{z$\leq$n} : \{n : $\N$\}                 $\ra$ \cc{zero}  \ct{$\leq$} n\\
\ \ \ \ \cc{s$\leq$s} : \{m n : $\N$\} $\ra$ (m$\leq$n : m \ct{$\leq$} n) $\ra$ \cc{suc} m \ct{$\leq$} \cc{suc} n

\end{block}

\begin{block}{Orden estricto entre naturales}
\cf{$\_<\_$} : $\N$ $\ra$ $\N$ $\ra$ \ct{Set}\\
m \cf{$<$} n = \cc{suc} m \cf{$\leq$} n
\end{block}

\begin{block}{head en Agda}
    \cf{head} : \{A : \ct{Set}\}  $\ra$ (xs : \ct{List} A) $\ra$ 
    \cc{zero} $<$ \ct{length} xs $\rightarrow$ A \\
    \cf{head} \cc{$[]$} p$_{[]}$ = ¿?\\
    \cf{head} (x \cc{$::$} xs) p$_{::}$ = x
  \end{block}  

\begin{block}{}
El tipo de p$_{::}$ : \cc{zero} $<$ \ct{length} (x \cc{$::$} xs)
\end{block}

\end{frame}

\begin{frame}

\begin{block}{Orden entre naturales}

\ck{data} \ct{\_$\leq$\_} : $\N$ $\ra$ $\N$ $\ra$ \ct{Set} \ck{where}\\
\ \ \ \ \cc{z$\leq$n} : \{n : $\N$\}                 $\ra$ \cc{zero}  \ct{$\leq$} n\\
\ \ \ \ \cc{s$\leq$s} : \{m n : $\N$\} $\ra$ (m$\leq$n : m \ct{$\leq$} n) $\ra$ \cc{suc} m \ct{$\leq$} \cc{suc} n

\end{block}

\begin{block}{Orden estricto entre naturales}
\cf{$\_<\_$} : $\N$ $\ra$ $\N$ $\ra$ \ct{Set}\\
m \cf{$<$} n = \cc{suc} m \cf{$\leq$} n
\end{block}

\begin{block}{head en Agda}
    \cf{head} : \{A : \ct{Set}\}  $\ra$ (xs : \ct{List} A) $\ra$ 
    \cc{zero} $<$ \ct{length} xs $\rightarrow$ A \\
    \cf{head} \cc{$[]$} p$_{[]}$ = ¿?\\
    \cf{head} (x \cc{$::$} xs) p$_{::}$ = x
  \end{block}  

\begin{block}{}
El tipo de p$_{::}$ : \cc{zero} $<$ \cc{suc} (\cf{length} xs)
\end{block}

\end{frame}

\begin{frame}

\begin{block}{Orden entre naturales}

\ck{data} \ct{\_$\leq$\_} : $\N$ $\ra$ $\N$ $\ra$ \ct{Set} \ck{where}\\
\ \ \ \ \cc{z$\leq$n} : \{n : $\N$\}                 $\ra$ \cc{zero}  \ct{$\leq$} n\\
\ \ \ \ \cc{s$\leq$s} : \{m n : $\N$\} $\ra$ (m$\leq$n : m \ct{$\leq$} n) $\ra$ \cc{suc} m \ct{$\leq$} \cc{suc} n

\end{block}

\begin{block}{Orden estricto entre naturales}
\cf{$\_<\_$} : $\N$ $\ra$ $\N$ $\ra$ \ct{Set}\\
m \cf{$<$} n = \cc{suc} m \cf{$\leq$} n
\end{block}

\begin{block}{head en Agda}
    \cf{head} : \{A : \ct{Set}\}  $\ra$ (xs : \ct{List} A) $\ra$ 
    \cc{zero} $<$ \ct{length} xs $\rightarrow$ A \\
    \cf{head} \cc{$[]$} p$_{[]}$ = ¿?\\
    \cf{head} (x \cc{$::$} xs) p$_{::}$ = x
  \end{block}  

\begin{block}{}
El tipo de p$_{::}$ : \cc{suc} \cc{zero} $\leq$ \cc{suc} (\cf{length} xs)
\end{block}

\end{frame}

\begin{frame}

\begin{block}{Orden entre naturales}

\ck{data} \ct{\_$\leq$\_} : $\N$ $\ra$ $\N$ $\ra$ \ct{Set} \ck{where}\\
\ \ \ \ \cc{z$\leq$n} : \{n : $\N$\}                 $\ra$ \cc{zero}  \ct{$\leq$} n\\
\ \ \ \ \cc{s$\leq$s} : \{m n : $\N$\} $\ra$ (m$\leq$n : m \ct{$\leq$} n) $\ra$ \cc{suc} m \ct{$\leq$} \cc{suc} n

\end{block}

\begin{block}{Orden estricto entre naturales}
\cf{$\_<\_$} : $\N$ $\ra$ $\N$ $\ra$ \ct{Set}\\
m \cf{$<$} n = \cc{suc} m \cf{$\leq$} n
\end{block}

\begin{block}{head en Agda}
    \cf{head} : \{A : \ct{Set}\}  $\ra$ (xs : \ct{List} A) $\ra$ 
    \cc{zero} $<$ \ct{length} xs $\rightarrow$ A \\
    \cf{head} \cc{$[]$} p$_{[]}$ = ¿?\\
    \cf{head} (x \cc{$::$} xs) p$_{::}$ = x
  \end{block}  

\begin{block}{}
El tipo de p$_{::}$ = \cc{s$\leq$s} p'\\
El tipo de p' : \cc{zero} $\leq$ \cf{length} xs
\end{block}

\end{frame}

\begin{frame}

\begin{block}{Orden entre naturales}

\ck{data} \ct{\_$\leq$\_} : $\N$ $\ra$ $\N$ $\ra$ \ct{Set} \ck{where}\\
\ \ \ \ \cc{z$\leq$n} : \{n : $\N$\}                 $\ra$ \cc{zero}  \ct{$\leq$} n\\
\ \ \ \ \cc{s$\leq$s} : \{m n : $\N$\} $\ra$ (m$\leq$n : m \ct{$\leq$} n) $\ra$ \cc{suc} m \ct{$\leq$} \cc{suc} n

\end{block}

\begin{block}{Orden estricto entre naturales}
\cf{$\_<\_$} : $\N$ $\ra$ $\N$ $\ra$ \ct{Set}\\
m \cf{$<$} n = \cc{suc} m \cf{$\leq$} n
\end{block}

\begin{block}{head en Agda}
    \cf{head} : \{A : \ct{Set}\}  $\ra$ (xs : \ct{List} A) $\ra$ 
    \cc{zero} $<$ \ct{length} xs $\rightarrow$ A \\
    \cf{head} \cc{$[]$} p$_{[]}$ = ¿?\\
    \cf{head} (x \cc{$::$} xs) (\cc{s$\leq$s} \cc{zero}) = x
  \end{block}  

\pause

\begin{block}{}
¿Podremos hacer lo mismo ahora para p$_{[]}$ ?\\ \pause
El tipo de p$_{[]}$ : \cc{zero} $<$ \ct{length} $[]$
\end{block}

\end{frame}

\begin{frame}

\begin{block}{Orden entre naturales}

\ck{data} \ct{\_$\leq$\_} : $\N$ $\ra$ $\N$ $\ra$ \ct{Set} \ck{where}\\
\ \ \ \ \cc{z$\leq$n} : \{n : $\N$\}                 $\ra$ \cc{zero}  \ct{$\leq$} n\\
\ \ \ \ \cc{s$\leq$s} : \{m n : $\N$\} $\ra$ (m$\leq$n : m \ct{$\leq$} n) $\ra$ \cc{suc} m \ct{$\leq$} \cc{suc} n

\end{block}

\begin{block}{Orden estricto entre naturales}
\cf{$\_<\_$} : $\N$ $\ra$ $\N$ $\ra$ \ct{Set}\\
m \cf{$<$} n = \cc{suc} m \cf{$\leq$} n
\end{block}

\begin{block}{head en Agda}
    \cf{head} : \{A : \ct{Set}\}  $\ra$ (xs : \ct{List} A) $\ra$ 
    \cc{zero} $<$ \ct{length} xs $\rightarrow$ A \\
    \cf{head} \cc{$[]$} p$_{[]}$ = ¿?\\
    \cf{head} (x \cc{$::$} xs) (\cc{s$\leq$s} \cc{zero}) = x
  \end{block}  

\begin{block}{}
El tipo de p$_{[]}$ : \cc{zero} $<$ \cc{zero}
\end{block}

\end{frame}

\begin{frame}

\begin{block}{Orden entre naturales}

\ck{data} \ct{\_$\leq$\_} : $\N$ $\ra$ $\N$ $\ra$ \ct{Set} \ck{where}\\
\ \ \ \ \cc{z$\leq$n} : \{n : $\N$\}                 $\ra$ \cc{zero}  \ct{$\leq$} n\\
\ \ \ \ \cc{s$\leq$s} : \{m n : $\N$\} $\ra$ (m$\leq$n : m \ct{$\leq$} n) $\ra$ \cc{suc} m \ct{$\leq$} \cc{suc} n

\end{block}

\begin{block}{Orden estricto entre naturales}
\cf{$\_<\_$} : $\N$ $\ra$ $\N$ $\ra$ \ct{Set}\\
m \cf{$<$} n = \cc{suc} m \cf{$\leq$} n
\end{block}

\begin{block}{head en Agda}
    \cf{head} : \{A : \ct{Set}\}  $\ra$ (xs : \ct{List} A) $\ra$ 
    \cc{zero} $<$ \ct{length} xs $\rightarrow$ A \\
    \cf{head} \cc{$[]$} p$_{[]}$ = ¿?\\
    \cf{head} (x \cc{$::$} xs) (\cc{s$\leq$s} \cc{zero}) = x
  \end{block}  

\begin{block}{}
El tipo de p$_{[]}$ : \cc{suc} \cc{zero} $\leq$ \cc{zero}
\end{block}

\end{frame}

\begin{frame}

\begin{block}{Orden entre naturales}

\ck{data} \ct{\_$\leq$\_} : $\N$ $\ra$ $\N$ $\ra$ \ct{Set} \ck{where}\\
\ \ \ \ \cc{z$\leq$n} : \{n : $\N$\}                 $\ra$ \cc{zero}  \ct{$\leq$} n\\
\ \ \ \ \cc{s$\leq$s} : \{m n : $\N$\} $\ra$ (m$\leq$n : m \ct{$\leq$} n) $\ra$ \cc{suc} m \ct{$\leq$} \cc{suc} n

\end{block}

\begin{block}{Orden estricto entre naturales}
\cf{$\_<\_$} : $\N$ $\ra$ $\N$ $\ra$ \ct{Set}\\
m \cf{$<$} n = \cc{suc} m \cf{$\leq$} n
\end{block}

\begin{block}{head en Agda}
    \cf{head} : \{A : \ct{Set}\}  $\ra$ (xs : \ct{List} A) $\ra$ 
    \cc{zero} $<$ \ct{length} xs $\rightarrow$ A \\
    \cf{head} \cc{$[]$} ()\\
    \cf{head} (x \cc{$::$} xs) (\cc{s$\leq$s} \cc{zero}) = x
\end{block}  

\end{frame}

\begin{frame}

  \begin{block}{head en Agda}
    \cf{head} : \{A : \ct{Set}\}  $\ra$ (xs : \ct{List} A) $\ra$ 
    \cc{zero} $<$ \ct{length} xs $\rightarrow$ A \\
    \cf{head} \cc{$[]$} ()\\
    \cf{head} (x \cc{$::$} xs) $\_$ = x
  \end{block}
  
\begin{block}{Pattern absurdo}

El pattern absurdo nos permite decir que un caso nunca puede
ocurrir cuando es imposible construir un valor del tipo requerido y por
lo tanto nos salvamos de tener que dar una definición de la función
para ese caso.

\end{block}

\end{frame}

\begin{frame}

\begin{block}{Replanteando el problema}
Las listas funcionan muy bien para muchas cosas. Pero tal vez
podamos crear un nuevo tipo de dato aprovechando el sistema de tipos
de Agda de manera tal de tener un tipo lista que no solo lleve la
información acerca del tipo de los elementos, sino además de su
tamaño.\\

De esta manera podemos definir un \cf{head} nuevo que no necesite
ninguna prueba para ser \textit{usado}.
\end{block}

\pause

\begin{block}{Vectores}

\ck{data} \ct{Vec} (A : \ct{Set}) : $\N$ $\ra$ \ct{Set} \ck{where}\\
\pause
\T \cc{$[]$}     : \ct{Vec} A \cc{zero}\\
\pause
\T \cc{\_$::$\_} : \{n : $\N$\} $\ra$ A $\ra$ \ct{Vec} A n $\ra$ \ct{Vec} A (\cc{suc} n)

\end{block}

\end{frame}

\begin{frame}

\begin{block}{Vectores}

\ck{data} \ct{Vec} (A : \ct{Set}) : $\N$ $\ra$ \ct{Set} \ck{where}\\
\T \cc{$[]$}     : \ct{Vec} A \cc{zero}\\
\T \cc{\_$::$\_} : \{n : $\N$\} $\ra$ A $\ra$ \ct{Vec} A n $\ra$ \ct{Vec} A (\cc{suc} n)

\end{block}

\begin{block}{Reimplementando el head}
Podemos pensar ahora cómo utilizar el tipo vector para realizar una
nueva implementación del \cf{head}
\end{block}

\pause

\begin{block}{head en Agda utilizando vectores}
    \cf{head} : \{A : \ct{Set}\} \{n : $\N$\} $\ra$ \ct{Vec} A (\cc{suc} n) $\ra$ A \\
    \pause
    \cf{head} (x \cc{$::$} xs) = x
\end{block}

\pause

\begin{block}{}
¿Donde quedó el caso de pattern matching para la lista vacia, \cc{$[]$}?
\end{block}

\end{frame}

\begin{frame}

  \begin{block}{}
  Aprovechando la implementación de los vectores podemos implementar distintas
  funciones nuevas.
  \end{block}

  \pause

  \begin{block}{}
  Recordemos la función \cf{filter}:
  \end{block}

  \begin{exampleblock}{función filter}
    \cf{filter} toma una colección de elementos $c$ de algún tipo A y un predicado sobre A, y retorna
    una colección con los elementos de $c$ que satisfacen el predicado.
  \end{exampleblock}

  \pause
  
  \begin{block}{}
    Queremos implementar \cf{filter} sobre el tipo de los vectores. Primero definamos una función
    que retorne la cantidad de elementos que satisfacen el predicado:
  \end{block}
  
\end{frame}

\begin{frame}

  \begin{block}{$\#$-filter}\small

  \cf{$\#$-filter} : \{A : \ct{Set}\} \{n : $\N$\} $\ra$ (A $\ra$ \ct{Bool}) $\ra$ \ct{Vec} A n $\ra$ $\N$\\
  \cf{$\#$-filter} p \cc{$[]$} $=$ \cc{zero}\\
  \cf{$\#$-filter} p (x \cc{$::$} xs) \ck{with} p x\\
  $\ldots$ $|$ \cc{true}  $=$ \cc{suc} (\cf{$\#$-filter} p xs)\\
  $\ldots$ $|$ \cc{false} $=$ \cf{$\#$-filter} p xs
  \end{block}
  
  \pause
  
  \begin{block}{}
    Luego podemos definir el filter:
  \end{block}

  \pause

  \begin{block}{Filter}\small

  \cf{filter} : \{A : \ct{Set}\} \{n : $\N$\} $\ra$\\
  \ \ \ \ \ \ \ \ \ (p : A $\ra$ \ct{Bool}) $\ra$ (xs : \ct{Vec} A n) $\ra$ \ct{Vec} A (\cf{$\#$-filter} p xs)\\
  \cf{filter} p \cc{$[]$} $=$ \cc{$[]$}\\
  \cf{filter} p (x \cc{$::$} xs) \ck{with} p x\\
  $\ldots$ $|$ \cc{true}  $=$ x \cc{$::$} (\cf{filter} p xs)\\
  $\ldots$ $|$ \cc{false} $=$ \cf{filter} p xs

  \end{block}

\end{frame}

\begin{frame}

\begin{block}{}
Ahora bien, se puede ver que calcular \cf{$\#$-filter} y \cf{filter} se
podría hacer al mismo tiempo. Solo necesitamos un tipo lo suficientemente
expresivo para poder construir valores, veamos como los pares dependientes
nos permiten realizar una implementación mejorada.
\end{block}

\pause

\begin{block}{Pares dependientes}\small

Constructor del tipo:\\ 
Dados tipos A : \ct{Set}, B : A $\ra$ \ct{Set} podemos construir el tipo
\ct{$\Sigma$[} x \ct{$\in$} A \ct{]} B\\
Constructor de valores: \cc{\_$,$\_}\\
Eliminadores: \\
\T - \cf{proj$_1$} : \ct{$\Sigma$[} x \ct{$\in$} A \ct{]} B $\ra$ A\\
\T - \cf{proj$_2$} : \ct{$\Sigma$[} x \ct{$\in$} A \ct{]} B $\ra$ B x\\

\end{block}

\pause

\begin{block}{}
 A este tipo también se lo puede escribir \ct{$\exists$} ($\lambda$ x $\ra$ B x)
\end{block}


\end{frame}

\begin{frame}

\begin{block}{Filter mejorado}\small

\cf{filterImproved} : \{A : \ct{Set}\} \{n : $\N$\} $\ra$\\
\T \T \T \T \ \ (A $\ra$ \ct{Bool}) $\ra$ (\ct{Vec} A n) $\ra$ \ct{$\exists$} ($\lambda$ m $\ra$ \ct{Vec} A m)\\
\pause
\cf{filterImproved} p \cc{$[]$} $=$ (\cc{zero} \cc{$,$} \cc{$[]$})\\
\cf{filterImproved} p (x \cc{$::$} xs) \ck{with} p x\\
$\ldots$ $|$ \cc{false} $=$ \cf{filterImproved} p xs\\
$\ldots$ $|$ \cc{true} $=$ \ck{let} (n' \cc{$,$} xs') $=$ \cf{filterImproved} p xs\\
\T \T \T  \ \ck{in} (\cc{suc} n' \cc{$,$} x \cc{$::$} xs')

\end{block}

\end{frame}


\begin{frame}

\begin{block}{}
  ¿Podremos especificar alguna propiedad más en el tipo resultante?
\end{block}

\pause

  \begin{block}{}
    Redefinamos el tipo del filter, agregando una ``postcondición'' más: El tamaño
    del vector resultante debe tener a lo sumo tantos elementos como el original
  \end{block}
  
  \pause

  \begin{block}{filter mejorado y con prueba}

  \cf{filterImproved'} : \{A : \ct{Set}\} \{n : $\N$\} $\ra$
  (A $\ra$ \ct{Bool}) $\ra$ (\ct{Vec} A n) $\ra$\\
  \T \T \T \T \ \ 
  \ct{$\exists$} ($\lambda$ m $\ra$ \ct{Vec} A m \ct{$\times$} m \ct{$\leq$} n)\\
  \pause
  \cf{filterImproved'} p \cc{$[]$} $=$ (\cc{zero} \cc{$,$} \cc{$[]$} \cc{$,$} \cc{z$\leq$n})\\
  \cf{filterImproved'} p (x \cc{$::$} xs) \ck{with} p x $|$ \cf{filterImproved'} p xs\\
  $\ldots$ $|$ \cc{false} $|$ (n' \cc{$,$} xs' \cc{$,$} m$\leq$n) = 
  (n' \cc{$,$} xs' \cc{$,$} \cf{prop$\leq$} m$\leq$n)\\
  $\ldots$ $|$ \cc{true}  $|$ (n' \cc{$,$} xs' \cc{$,$} m$\leq$n) = 
  (\cc{suc} n' , x \cc{$::$} xs' , \cc{s$\leq$s} m$\leq$n)

  \end{block}
  
\end{frame}

\begin{frame}

\begin{block}{}
  Creemos que esta implementación es correcta, pero el tipo de la función no asegura que cumpla
  con la especificación que detallamos.
  
  Podemos redefinirla con el siguiente tipo:
\end{block}

\begin{block}{Filter con especificación}
  Lo mostramos en Agda.
\end{block}

\end{frame}

\end{document}

