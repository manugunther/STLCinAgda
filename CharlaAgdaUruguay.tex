\documentclass[xcolor=dvipsnames]{beamer} % en min\'uscula!
\usefonttheme{professionalfonts} % fuentes de LaTeX
\usetheme{Warsaw} % tema escogido en este ejemplo
\usecolortheme[named=blue]{structure}
%%%% packages y comandos personales %%%%
\usepackage[utf8]{inputenc}
\usepackage{latexsym,amsmath,amssymb} % S\'imbolos
% \usepackage{xcolor}
\usepackage{MnSymbol}


\usepackage{bussproofs} 

\newcommand{\cf}[1]{\textcolor{blue}{#1}}
\newcommand{\ct}[1]{\textcolor{blue}{#1}}
\newcommand{\cc}[1]{\textcolor{ForestGreen}{#1}}
\newcommand{\ck}[1]{\textcolor{orange}{#1}}

\newcommand{\N}{\ct{\mathbb{N}}}
\newcommand{\ra}{\rightarrow}
\newcommand{\T}{ \ \ \ \ }

 \newcommand{\tjud}[2]
  {\ensuremath{#1\,:\,#2}}
 \newcommand{\depFun}[3] {
  {\ensuremath{\Pi_{(\tjud{#1}{#2})}\,#3\,#1}}
 }
 \newcommand{\cprod}[2]
  {\ensuremath{#1 \times #2}
 }
 \newcommand{\depPair}[3]
 {\ensuremath{\Sigma_{(\tjud{#1}{#2})}\,#3\,#1}
 }

\begin{document}
\beamertemplatenavigationsymbolsempty
\title{Agda, un lenguaje con tipos dependientes}
\author{Alejandro Gadea, Emmanuel Gunther}
\institute{FaMAF, Universidad Nacional de Córdoba}
\date{14 de Noviembre de 2014}
\frame{\titlepage}

\section{Introducción}

\begin{frame}

\begin{block}{ }
 En el desarrollo de software los programas se escriben esperando que
 satisfaga una especificación.
\end{block}

\pause

\begin{exampleblock}{Especificación de head}
 La función \cf{head} toma una lista de elementos no vacía y retorna el primero de ellos.
\end{exampleblock}

\pause

\begin{block}{}
 En la práctica en general no se asegura que un programa cumpla con la especificación.
\end{block}


\end{frame}

\begin{frame}

\begin{block}{ }
 Los sistemas de tipos agregan expresividad a los programas, permitiendo expresar propiedades que deben satisfacer.
\end{block}

\pause

\begin{exampleblock}{head en python}
  \ck{def} \cf{head} (l):
  
  \T \ck{return} l[0]
\end{exampleblock}

\pause

\begin{block}{}
  Podríamos tener en un programa la línea \cf{head} (\cc{true}) (lo cual no cumple una precondición) 
  y tendríamos un error en tiempo de ejecución.
\end{block}

\end{frame}

\begin{frame}
  \begin{block}{}
   En un lenguaje con sistema de tipos fuerte podemos especificar que la función \cf{head} 
   toma como argumento una lista y retorna un elemento:
  \end{block}

  \pause
  
  \begin{exampleblock}{head en haskell}
    \cf{head} :: [a] $\rightarrow$ a
    
    \cf{head} (x : xs) = x
  \end{exampleblock}  

  \pause
  
  \begin{block}{}
    Sin embargo podríamos hacer \cf{head} $[]$, lo cual violaría la precondición de que la lista no sea vacía.
     y tendríamos un error en tiempo de ejecución.
  \end{block}
  

\end{frame}

\begin{frame}
  \begin{block}{}
    Fortaleciendo el sistema de tipos podemos agregar información sobre lo que se espera de un programa:
  \end{block}

  \pause
  
  \begin{exampleblock}{head en agda}
    \cf{head} : \{a : Set\}  $\rightarrow$ (xs : List a) $\rightarrow$ 0 $<$ length xs $\rightarrow$ a \\
    \cf{head} $[]$ ()\\
    \cf{head} (x $ :: $ xs) \_ = x
  \end{exampleblock}  

  \pause
  
  \begin{block}{}
    La función \cf{head} ahora espera una lista de elementos y una \textit{prueba} de que tiene más de un elemento.
  \end{block}

  \pause
  
  \begin{block}{}
    Nunca podría llamarse \cf{head} con una lista vacía ya que no se podría \textit{construir} 0 $<$ length $[]$.
  \end{block}

\end{frame}

\section{Tipos de datos y proposiciones lógicas}

\begin{frame}
  \begin{block}{}
    A los tipos de datos que tenemos en lenguajes como Haskell, se los puede fortalecer teniendo tipos que \textbf{dependen}
    de valores.
  \end{block}
  
  \pause
  
  \begin{block}{Familia de tipos}
    Si $A$ es un tipo de dato, podemos tener una función $B$ que toma un elemento de $A$ y retorna un tipo.
    A una función así se la denomina \textbf{familia de tipos} indexada en $A$.
  \end{block}  

  \pause
    
  \begin{block}{}
    Con este sistema de tipos más fuerte, se puede ver una correspondencia entre las proposiciones lógicas
    y los tipos de datos.
  \end{block}

\end{frame}

\begin{frame}

   \begin{block}{Reglas para el $\wedge$}
      \AxiomC{$p$}
      \AxiomC{$q$}
      \LeftLabel{Introducción de $\wedge$:}
      \BinaryInfC{$p \wedge q$}
      \DisplayProof
      \quad
      \AxiomC{$p \wedge q$}
      \LeftLabel{Eliminación de $\wedge$:}
      \UnaryInfC{$p$}
      \DisplayProof
      \quad
      \AxiomC{$p \wedge q$}
      \UnaryInfC{$q$}
      \DisplayProof
    \end{block}
    
    \pause
    
    \begin{block}{Reglas del tipo producto (tuplas en Haskell)}
      \AxiomC{$\tjud{a}{A}$}
      \AxiomC{$\tjud{b}{B}$}
      \BinaryInfC{$\tjud{(a,b)}{\cprod{A}{B}}$}
      \DisplayProof
      \quad
      \AxiomC{$\tjud{t}{\cprod{A}{B}}$}
      \UnaryInfC{$\tjud{pr_1\,t}{A}$}
      \DisplayProof
      \quad
      \AxiomC{$\tjud{t}{\cprod{A}{B}}$}
      \UnaryInfC{$\tjud{pr_2\,t}{B}$}
      \DisplayProof
    \end{block}


\end{frame}

\begin{frame}

 \begin{block}{}
  Podemos hacer la misma correspondencia con el resto de los conectivos y otros tipos de datos:
 \end{block}

 \pause
 
 \begin{block}{Lógica proposicional y teoría de tipos}
  \begin{itemize}
  \item $\mathbf{True}$ se corresponde con $\mathbf{\top}$ (tipo con exactamente \textbf{un} constructor).
  \item $\mathbf{False}$ se corresponde con $\mathbf{\bot}$ (tipo sin constructores).
  \item $\mathbf{A \wedge B}$ se corresponde con $\cprod{A}{B}$ (tipo de 2-uplas).
  \item $\mathbf{A \vee B}$ se corresponde con $A + B$ (tipo de la unión disjunta).
  \item $\mathbf{A \Rightarrow B}$ se corresponde con $A \rightarrow B$ (funciones de $A$ en $B$).
  \item $\mathbf{A \Leftrightarrow B}$ se corresponde con $\cprod{(A \rightarrow B)}{(B \rightarrow A)}$.
  \item $\mathbf{\neg A}$ se corresponde con $A \rightarrow \mathbf{\bot}$.
\end{itemize}
 \end{block}
 
\end{frame}

\begin{frame}
     \begin{block}{}
      Si pensamos un tipo como una proposición lógica, para probar su validez debemos \textbf{construir un elemento}
      del tipo.
     \end{block}
     
     \pause
     
     \begin{block}{}
      El tipo $\top$ tiene un solo constructor ($\tjud{tt}{\top}$). Por lo tanto \textbf{siempre podemos
      construir} un elemento de $\top$.
      
      Análogamente, $\mathbf{True}$ es \textbf{siempre válido}.
     \end{block}

     \pause
     
      \begin{block}{}
      El tipo $\bot$ no tiene constuctores. Por lo tanto \textbf{nunca podemos construir} un elemento
      de $\bot$
      
      Análogamente, $\mathbf{False}$ \textbf{nunca puede probarse}.
     \end{block}

\end{frame}

\begin{frame}
 \begin{block}{}
  Con la posibilidad de definir familias de tipos podemos definir \textbf{funciones dependientes} y \textbf{pares dependientes}.
 \end{block}
 
 \pause
 
 \begin{block}{}
  Si $A$ es un tipo y $B$ una familia indexada en $A$, tenemos el tipo de funciones dependientes:
  $(\tjud{x}{A}) \rightarrow B\,x$.
 \end{block}
 
 \pause
 
 \begin{block}{}
  Si $A$ es un tipo y $B$ una familia indexada en $A$, tenemos el tipo de pares dependientes:
  $\depPair{x}{A}{B}$. Los elementos de este tipo son pares donde el tipo del segundo elemento \textbf{depende}
  del valor del primero.
 \end{block} 
 
\end{frame}


\begin{frame}
 
   \begin{block}{}
  Con estos tipos más complejos, podemos tener una correspondencia con los cuantificadores del cálculo
  de predicados, completando la tabla anterior:
 \end{block}

 \pause

  \begin{block}{Lógica de predicados y teoría de tipos}
   \begin{itemize}
    \item ($\forall x$, $P\,x$) se corresponde con $(\tjud{x}{A}) \rightarrow P\,x$.
    \item ($\exists x$, $P\,x$) se corresponde con $\depPair{x}{A}{P}$.
    \end{itemize}
  \end{block}

  \pause
  
  \begin{block}{}
   Gracias a esta correspondencia podemos tener un \textbf{lenguaje de programación} con un sistema
   de tipos que \textbf{permita expresar proposiciones lógicas}.
  \end{block}

\end{frame}


\section{Introducción a Agda}

\begin{frame}

\begin{block}{}
Implementando la función \cf{head} segura, veremos algunas características del lenguaje \textbf{Agda}, el
cual implementa un sistema de tipos con las propiedades que mencionamos.
\end{block}

\pause

\begin{block}{Listas de A}

\ck{data} \ct{List} (A : \ct{Set}) : \ct{Set} \ck{where}\\
\T \cc{$[]$} : \ct{List} A\\
\T \cc{$\_::\_$} : A $\ra$ \ct{List} A $\ra$ \ct{List} A

\end{block}

\pause

\begin{block}{Naturales}

\ck{data} $\N$ : \ct{Set} \ck{where}\\
\T \cc{zero} : $\N$\\
\T \cc{suc}  : $\N$ $\ra$ $\N$

\end{block}

\end{frame}

\begin{frame}
\begin{block}{}
Recordando el tipo del \cf{head}:\\

\
\{A : \ct{Set}\} $\ra$ 
(xs : \ct{List} A) $\ra$ \cc{zero} $<$ \cf{length} xs $\ra$ A\\

\

Nos falta definir el tipo $<$.
\end{block}

\pause

\begin{block}{Orden entre naturales}

\ck{data} \ct{\_$\leq$\_} : $\N$ $\ra$ $\N$ $\ra$ \ct{Set} \ck{where}\\
\ \ \ \ \cc{z$\leq$n} : \{n : $\N$\}                 $\ra$ \cc{zero}  \ct{$\leq$} n\\
\ \ \ \ \cc{s$\leq$s} : \{m n : $\N$\} $\ra$ (m$\leq$n : m \ct{$\leq$} n) $\ra$ \cc{suc} m \ct{$\leq$} \cc{suc} n

\end{block}

\pause

\begin{block}{Orden estricto entre naturales}
\cf{$\_<\_$} : $\N$ $\ra$ $\N$ $\ra$ \ct{Set}\\
m \cf{$<$} n = \cc{suc} m \cf{$\leq$} n
\end{block}

\pause

\begin{block}{}
 Observemos que estos tipos representan proposiciones lógicas. Los valores serán pruebas de su validez.
\end{block}


\end{frame}

\begin{frame}[shrink=5]

\begin{block}{Orden entre naturales}

\ck{data} \ct{\_$\leq$\_} : $\N$ $\ra$ $\N$ $\ra$ \ct{Set} \ck{where}\\
\ \ \ \ \cc{z$\leq$n} : \{n : $\N$\}                 $\ra$ \cc{zero}  \ct{$\leq$} n\\
\ \ \ \ \cc{s$\leq$s} : \{m n : $\N$\} $\ra$ (m$\leq$n : m \ct{$\leq$} n) $\ra$ \cc{suc} m \ct{$\leq$} \cc{suc} n

\end{block}

\begin{block}{Orden estricto entre naturales}
\cf{$\_<\_$} : $\N$ $\ra$ $\N$ $\ra$ \ct{Set}\\
m \cf{$<$} n = \cc{suc} m \cf{$\leq$} n
\end{block}

\begin{block}{head en Agda}
    \cf{head} : \{A : \ct{Set}\}  $\ra$ (xs : \ct{List} A) $\ra$ 
    \cc{zero} $<$ \ct{length} xs $\rightarrow$ A \\
    \pause
    \cf{head} \cc{$[]$} p$_{[]}$ = ¿?\\
    \cf{head} (x \cc{$::$} xs) p$_{::}$ = ¿?
\end{block}  

\begin{block}{}
En Agda todas las funciones deben
ser \textbf{totales}. 

Debemos considerar los dos casos de pattern matching: \cc{$[]$} y \cc{$::$}.
\end{block}

\end{frame}

\begin{frame}

\begin{block}{Orden entre naturales}

\ck{data} \ct{\_$\leq$\_} : $\N$ $\ra$ $\N$ $\ra$ \ct{Set} \ck{where}\\
\ \ \ \ \cc{z$\leq$n} : \{n : $\N$\}                 $\ra$ \cc{zero}  \ct{$\leq$} n\\
\ \ \ \ \cc{s$\leq$s} : \{m n : $\N$\} $\ra$ (m$\leq$n : m \ct{$\leq$} n) $\ra$ \cc{suc} m \ct{$\leq$} \cc{suc} n

\end{block}

\begin{block}{Orden estricto entre naturales}
\cf{$\_<\_$} : $\N$ $\ra$ $\N$ $\ra$ \ct{Set}\\
m \cf{$<$} n = \cc{suc} m \cf{$\leq$} n
\end{block}

\begin{block}{head en Agda}
    \cf{head} : \{A : \ct{Set}\}  $\ra$ (xs : \ct{List} A) $\ra$ 
    \cc{zero} $<$ \ct{length} xs $\rightarrow$ A \\
    \cf{head} \cc{$[]$} p$_{[]}$ = ¿?\\
    \cf{head} (x \cc{$::$} xs) p$_{::}$ = x
  \end{block}  

\pause

\begin{block}{}
Podemos preguntarnos que valores (o pruebas) pueden ser p$_{[]}$ y p$_{::}$
\end{block}

\end{frame}

\begin{frame}[shrink=5]

\begin{block}{Orden entre naturales}

\ck{data} \ct{\_$\leq$\_} : $\N$ $\ra$ $\N$ $\ra$ \ct{Set} \ck{where}\\
\ \ \ \ \cc{z$\leq$n} : \{n : $\N$\}                 $\ra$ \cc{zero}  \ct{$\leq$} n\\
\ \ \ \ \cc{s$\leq$s} : \{m n : $\N$\} $\ra$ (m$\leq$n : m \ct{$\leq$} n) $\ra$ \cc{suc} m \ct{$\leq$} \cc{suc} n

\end{block}

\begin{block}{Orden estricto entre naturales}
\cf{$\_<\_$} : $\N$ $\ra$ $\N$ $\ra$ \ct{Set}\\
m \cf{$<$} n = \cc{suc} m \cf{$\leq$} n
\end{block}

\begin{block}{head en Agda}
    \cf{head} : \{A : \ct{Set}\}  $\ra$ (xs : \ct{List} A) $\ra$ 
    \cc{zero} $<$ \ct{length} xs $\rightarrow$ A \\
    \cf{head} \cc{$[]$} p$_{[]}$ = ¿?\\
    \cf{head} (x \cc{$::$} xs) p$_{::}$ = x
  \end{block}  

\begin{block}{}
El tipo de p$_{::}$ : \cc{zero} $<$ \ct{length} (x \cc{$::$} xs) \pause
(Notar que como podemos tener valores en los tipos ocurre que para que estos tipos
sean chequeados tenemos que \textit{computar} estos valores.)
\end{block}

\end{frame}

\begin{frame}

\begin{block}{Orden entre naturales}

\ck{data} \ct{\_$\leq$\_} : $\N$ $\ra$ $\N$ $\ra$ \ct{Set} \ck{where}\\
\ \ \ \ \cc{z$\leq$n} : \{n : $\N$\}                 $\ra$ \cc{zero}  \ct{$\leq$} n\\
\ \ \ \ \cc{s$\leq$s} : \{m n : $\N$\} $\ra$ (m$\leq$n : m \ct{$\leq$} n) $\ra$ \cc{suc} m \ct{$\leq$} \cc{suc} n

\end{block}

\begin{block}{Orden estricto entre naturales}
\cf{$\_<\_$} : $\N$ $\ra$ $\N$ $\ra$ \ct{Set}\\
m \cf{$<$} n = \cc{suc} m \cf{$\leq$} n
\end{block}

\begin{block}{head en Agda}
    \cf{head} : \{A : \ct{Set}\}  $\ra$ (xs : \ct{List} A) $\ra$ 
    \cc{zero} $<$ \ct{length} xs $\rightarrow$ A \\
    \cf{head} \cc{$[]$} p$_{[]}$ = ¿?\\
    \cf{head} (x \cc{$::$} xs) p$_{::}$ = x
  \end{block}  

\begin{block}{}
El tipo de p$_{::}$ : \cc{zero} $<$ \cc{suc} (\cf{length} xs)
\end{block}

\end{frame}

\begin{frame}

\begin{block}{Orden entre naturales}

\ck{data} \ct{\_$\leq$\_} : $\N$ $\ra$ $\N$ $\ra$ \ct{Set} \ck{where}\\
\ \ \ \ \cc{z$\leq$n} : \{n : $\N$\}                 $\ra$ \cc{zero}  \ct{$\leq$} n\\
\ \ \ \ \cc{s$\leq$s} : \{m n : $\N$\} $\ra$ (m$\leq$n : m \ct{$\leq$} n) $\ra$ \cc{suc} m \ct{$\leq$} \cc{suc} n

\end{block}

\begin{block}{Orden estricto entre naturales}
\cf{$\_<\_$} : $\N$ $\ra$ $\N$ $\ra$ \ct{Set}\\
m \cf{$<$} n = \cc{suc} m \cf{$\leq$} n
\end{block}

\begin{block}{head en Agda}
    \cf{head} : \{A : \ct{Set}\}  $\ra$ (xs : \ct{List} A) $\ra$ 
    \cc{zero} $<$ \ct{length} xs $\rightarrow$ A \\
    \cf{head} \cc{$[]$} p$_{[]}$ = ¿?\\
    \cf{head} (x \cc{$::$} xs) p$_{::}$ = x
  \end{block}  

\begin{block}{}
El tipo de p$_{::}$ : \cc{suc} \cc{zero} $\leq$ \cc{suc} (\cf{length} xs)
\end{block}

\end{frame}

\begin{frame}

\begin{block}{Orden entre naturales}

\ck{data} \ct{\_$\leq$\_} : $\N$ $\ra$ $\N$ $\ra$ \ct{Set} \ck{where}\\
\ \ \ \ \cc{z$\leq$n} : \{n : $\N$\}                 $\ra$ \cc{zero}  \ct{$\leq$} n\\
\ \ \ \ \cc{s$\leq$s} : \{m n : $\N$\} $\ra$ (m$\leq$n : m \ct{$\leq$} n) $\ra$ \cc{suc} m \ct{$\leq$} \cc{suc} n

\end{block}

\begin{block}{Orden estricto entre naturales}
\cf{$\_<\_$} : $\N$ $\ra$ $\N$ $\ra$ \ct{Set}\\
m \cf{$<$} n = \cc{suc} m \cf{$\leq$} n
\end{block}

\begin{block}{head en Agda}
    \cf{head} : \{A : \ct{Set}\}  $\ra$ (xs : \ct{List} A) $\ra$ 
    \cc{zero} $<$ \ct{length} xs $\rightarrow$ A \\
    \cf{head} \cc{$[]$} p$_{[]}$ = ¿?\\
    \cf{head} (x \cc{$::$} xs) p$_{::}$ = x
  \end{block}  

\begin{block}{}
El valor de p$_{::}$ = \cc{s$\leq$s} p'\\
El tipo de p' : \cc{zero} $\leq$ \cf{length} xs
\end{block}

\end{frame}

\begin{frame}

\begin{block}{Orden entre naturales}

\ck{data} \ct{\_$\leq$\_} : $\N$ $\ra$ $\N$ $\ra$ \ct{Set} \ck{where}\\
\ \ \ \ \cc{z$\leq$n} : \{n : $\N$\}                 $\ra$ \cc{zero}  \ct{$\leq$} n\\
\ \ \ \ \cc{s$\leq$s} : \{m n : $\N$\} $\ra$ (m$\leq$n : m \ct{$\leq$} n) $\ra$ \cc{suc} m \ct{$\leq$} \cc{suc} n

\end{block}

\begin{block}{Orden estricto entre naturales}
\cf{$\_<\_$} : $\N$ $\ra$ $\N$ $\ra$ \ct{Set}\\
m \cf{$<$} n = \cc{suc} m \cf{$\leq$} n
\end{block}

\begin{block}{head en Agda}
    \cf{head} : \{A : \ct{Set}\}  $\ra$ (xs : \ct{List} A) $\ra$ 
    \cc{zero} $<$ \ct{length} xs $\rightarrow$ A \\
    \cf{head} \cc{$[]$} p$_{[]}$ = ¿?\\
    \cf{head} (x \cc{$::$} xs) p$_{::}$ = x
  \end{block}  

\begin{block}{}
El valor de p$_{::}$ = \cc{s$\leq$s} p'\\
El valor de p' = \cc{z$\leq$n}
\end{block}

\end{frame}

\begin{frame}

\begin{block}{Orden entre naturales}

\ck{data} \ct{\_$\leq$\_} : $\N$ $\ra$ $\N$ $\ra$ \ct{Set} \ck{where}\\
\ \ \ \ \cc{z$\leq$n} : \{n : $\N$\}                 $\ra$ \cc{zero}  \ct{$\leq$} n\\
\ \ \ \ \cc{s$\leq$s} : \{m n : $\N$\} $\ra$ (m$\leq$n : m \ct{$\leq$} n) $\ra$ \cc{suc} m \ct{$\leq$} \cc{suc} n

\end{block}

\begin{block}{Orden estricto entre naturales}
\cf{$\_<\_$} : $\N$ $\ra$ $\N$ $\ra$ \ct{Set}\\
m \cf{$<$} n = \cc{suc} m \cf{$\leq$} n
\end{block}

\begin{block}{head en Agda}
    \cf{head} : \{A : \ct{Set}\}  $\ra$ (xs : \ct{List} A) $\ra$ 
    \cc{zero} $<$ \ct{length} xs $\rightarrow$ A \\
    \cf{head} \cc{$[]$} p$_{[]}$ = ¿?\\
    \cf{head} (x \cc{$::$} xs) (\cc{s$\leq$s} \cc{z$\leq$n}) = x
  \end{block}  

\pause

\begin{block}{}
¿Podremos hacer lo mismo ahora para p$_{[]}$ ?\\ \pause
El tipo de p$_{[]}$ : \cc{zero} $<$ \ct{length} $[]$
\end{block}

\end{frame}

\begin{frame}

\begin{block}{Orden entre naturales}

\ck{data} \ct{\_$\leq$\_} : $\N$ $\ra$ $\N$ $\ra$ \ct{Set} \ck{where}\\
\ \ \ \ \cc{z$\leq$n} : \{n : $\N$\}                 $\ra$ \cc{zero}  \ct{$\leq$} n\\
\ \ \ \ \cc{s$\leq$s} : \{m n : $\N$\} $\ra$ (m$\leq$n : m \ct{$\leq$} n) $\ra$ \cc{suc} m \ct{$\leq$} \cc{suc} n

\end{block}

\begin{block}{Orden estricto entre naturales}
\cf{$\_<\_$} : $\N$ $\ra$ $\N$ $\ra$ \ct{Set}\\
m \cf{$<$} n = \cc{suc} m \cf{$\leq$} n
\end{block}

\begin{block}{head en Agda}
    \cf{head} : \{A : \ct{Set}\}  $\ra$ (xs : \ct{List} A) $\ra$ 
    \cc{zero} $<$ \ct{length} xs $\rightarrow$ A \\
    \cf{head} \cc{$[]$} p$_{[]}$ = ¿?\\
    \cf{head} (x \cc{$::$} xs) (\cc{s$\leq$s} \cc{z$\leq$n}) = x
  \end{block}  

\begin{block}{}
El tipo de p$_{[]}$ : \cc{zero} $<$ \cc{zero}
\end{block}

\end{frame}

\begin{frame}

\begin{block}{Orden entre naturales}

\ck{data} \ct{\_$\leq$\_} : $\N$ $\ra$ $\N$ $\ra$ \ct{Set} \ck{where}\\
\ \ \ \ \cc{z$\leq$n} : \{n : $\N$\}                 $\ra$ \cc{zero}  \ct{$\leq$} n\\
\ \ \ \ \cc{s$\leq$s} : \{m n : $\N$\} $\ra$ (m$\leq$n : m \ct{$\leq$} n) $\ra$ \cc{suc} m \ct{$\leq$} \cc{suc} n

\end{block}

\begin{block}{Orden estricto entre naturales}
\cf{$\_<\_$} : $\N$ $\ra$ $\N$ $\ra$ \ct{Set}\\
m \cf{$<$} n = \cc{suc} m \cf{$\leq$} n
\end{block}

\begin{block}{head en Agda}
    \cf{head} : \{A : \ct{Set}\}  $\ra$ (xs : \ct{List} A) $\ra$ 
    \cc{zero} $<$ \ct{length} xs $\rightarrow$ A \\
    \cf{head} \cc{$[]$} p$_{[]}$ = ¿?\\
    \cf{head} (x \cc{$::$} xs) (\cc{s$\leq$s} \cc{z$\leq$n}) = x
  \end{block}  

\begin{block}{}
El tipo de p$_{[]}$ : \cc{suc} \cc{zero} $\leq$ \cc{zero} \pause (pero esta proposición es falsa)
\end{block}

\end{frame}

\begin{frame}

\begin{block}{Orden entre naturales}

\ck{data} \ct{\_$\leq$\_} : $\N$ $\ra$ $\N$ $\ra$ \ct{Set} \ck{where}\\
\ \ \ \ \cc{z$\leq$n} : \{n : $\N$\}                 $\ra$ \cc{zero}  \ct{$\leq$} n\\
\ \ \ \ \cc{s$\leq$s} : \{m n : $\N$\} $\ra$ (m$\leq$n : m \ct{$\leq$} n) $\ra$ \cc{suc} m \ct{$\leq$} \cc{suc} n

\end{block}

\begin{block}{Orden estricto entre naturales}
\cf{$\_<\_$} : $\N$ $\ra$ $\N$ $\ra$ \ct{Set}\\
m \cf{$<$} n = \cc{suc} m \cf{$\leq$} n
\end{block}

\begin{block}{head en Agda}
    \cf{head} : \{A : \ct{Set}\}  $\ra$ (xs : \ct{List} A) $\ra$ 
    \cc{zero} $<$ \ct{length} xs $\rightarrow$ A \\
    \cf{head} \cc{$[]$} ()\\
    \cf{head} (x \cc{$::$} xs) (\cc{s$\leq$s} \cc{z$\leq$n}) = x
\end{block}  

\begin{block}{}
 Es imposible obtener un elemento de \cc{suc} \cc{zero} $\leq$ \cc{zero} a partir de los constructores
 del tipo $\leq$.
\end{block}


\end{frame}

\begin{frame}

  \begin{block}{head en Agda}
    \cf{head} : \{A : \ct{Set}\}  $\ra$ (xs : \ct{List} A) $\ra$ 
    \cc{zero} $<$ \ct{length} xs $\rightarrow$ A \\
    \cf{head} \cc{$[]$} ()\\
    \cf{head} (x \cc{$::$} xs) $\_$ = x
  \end{block}
  
\begin{block}{Pattern absurdo}

Cuando un caso del pattern matching nunca puede ocurrir ya que es imposible
construir un valor del tipo requerido tenemos un \textbf{pattern absurdo}.


\end{block}

\end{frame}

\begin{frame}

\begin{block}{Replanteando el problema}

En vez de tener listas de cualquier tamaño y asegurar
que no son vacías para usar la función \cf{head}, podríamos
tener listas de tamaño fijo, a las que llamamos \textbf{vectores}.
\end{block}

\pause

\begin{block}{Vectores}

\ck{data} \ct{Vec} (A : \ct{Set}) : $\N$ $\ra$ \ct{Set} \ck{where}\\
\pause
\T \cc{$[]$}     : \ct{Vec} A \cc{zero}\\
\pause
\T \cc{\_$::$\_} : \{n : $\N$\} $\ra$ A $\ra$ \ct{Vec} A n $\ra$ \ct{Vec} A (\cc{suc} n)

\end{block}

\pause

\begin{block}{head en Agda utilizando vectores}
    \cf{head} : \{A : \ct{Set}\} \{n : $\N$\} $\ra$ \ct{Vec} A (\cc{suc} n) $\ra$ A \\
    \pause
    \cf{head} (x \cc{$::$} xs) = x
\end{block}


\pause

\begin{block}{}
¿Donde quedó el caso de pattern matching para la lista vacia, \cc{$[]$}?
\end{block}


\end{frame}


\begin{frame}
 
 \begin{block}{Conclusiones}
  \begin{itemize}
  \pause
  
   \item Sistemas de tipos fuertes agregan expresividad a los lenguajes de programación
    permitiendo eliminar errores en etapas tempranas del desarrollo.
    
   \pause
    
   \item \textbf{Agda} implementa un sistema de tipos que permite tener una analogía entre la lógica y los 
   tipos de datos.
   
   \pause
   
   \item El chequeo de tipos en estos lenguajes es más complejo ya que requiere normalizar expresiones.
   
   \pause
   
   \item Mediante el tipo de un programa se pueden expresar pre y post condiciones, como en el ejemplo
   del \cf{head}.
  \end{itemize}

 \end{block}

 \pause
 
 \begin{block}{}
  Como ejemplo de un programa no trivial certificado, implementamos inferencia de tipos para una versión del cálculo
  lambda simplemente tipado en Agda.
 \end{block}

 
\end{frame}

\begin{frame}{}
\begin{center}
 \Huge Fin.
\end{center}
\end{frame}


\end{document}
