\documentclass[xcolor=dvipsnames]{beamer} % en min\'uscula!
\usefonttheme{professionalfonts} % fuentes de LaTeX
\usetheme{Warsaw} % tema escogido en este ejemplo
\usecolortheme[named=blue]{structure}
%%%% packages y comandos personales %%%%
\usepackage[latin1]{inputenc}
\usepackage{latexsym,amsmath,amssymb} % S\'imbolos
\usepackage{xcolor}
\usepackage{MnSymbol}

\newcommand{\cf}[1]{\textcolor{blue}{#1}}
\newcommand{\ct}[1]{\textcolor{blue}{#1}}
\newcommand{\cc}[1]{\textcolor{ForestGreen}{#1}}
\newcommand{\ck}[1]{\textcolor{orange}{#1}}

\newcommand{\N}{\ct{\mathbb{N}}}
\newcommand{\ra}{\rightarrow}
\newcommand{\T}{ \ \ \ \ }

\begin{document}
\beamertemplatenavigationsymbolsempty
\title{Creando un mundo mejor con Agda}
\author{{Alejandro Gadea, Emmanuel Gunther}\\
\vspace*{0.5cm}}
\date{14 de Noviembre de 2014}
\frame{\titlepage}

\section{Introducci�n}


\begin{frame}

- Presentar las motivaciones para la teor�a de tipos dependientes,
  usando el ejemplo del head
- Resumen teor�a de tipos, mencionando la correspondencia con la l�gica.
- Agda
- 
  

\end{frame}

\begin{frame}
\frametitle{Introducci�n a Agda}
\framesubtitle{Tipos de datos}

\begin{block}{Naturales}

\ck{data} $\N$ : \ct{Set} \ck{where}\\
\ \ \ \ \cc{zero} : $\N$\\
\ \ \ \ \cc{suc}  : $\N$ $\ra$ $\N$

\end{block}

\pause

\begin{block}{Suma de naturales}

\cf{\_$+$\_} : $\N$ $\ra$ $\N$ $\ra$ $\N$\\
\cc{zero}  \cf{$+$} n $=$ n\\
\cc{suc} m \cf{$+$} n $=$ \cc{suc} (m \cf{$+$} n)

\end{block}

\end{frame}

\begin{frame}
\frametitle{Introducci�n a Agda}
\framesubtitle{Tipos de datos}

\begin{block}{Vectores}
\ck{data} \ct{Vec} (A : \ct{Set}) : $\N$ $\ra$ \ct{Set} \ck{where}\\
\pause
\T \cc{$[]$}     : \ct{Vec} A \cc{zero}\\
\pause
\T \cc{\_$::$\_} : \{n : $\N$\} $\ra$ A $\ra$ \ct{Vec} A n $\ra$ \ct{Vec} A (\cc{suc} n)
\end{block}

\pause

\begin{block}{Producto}

\ck{data} \ct{\_$\times$\_} (A B : \ct{Set}) : \ct{Set} \ck{where}\\
\ \ \ \ \cc{\_$,$\_} : A $\ra$ B $\ra$ A \ct{$\times$} B

\end{block}

\pause

\begin{block}{Zip}

\cf{zip} : \{A B : \ct{Set}\} \{n : $\N$\} $\ra$ 
\ct{Vec} A n $\ra$ \ct{Vec} B n $\ra$ \ct{Vec} (A \ct{$\times$} B) n\\
\cf{zip} \cc{$[]$} \cc{$[]$} $=$ \cc{$[]$}\\
\cf{zip} (a \cc{$::$} vas) (b \cc{$::$} vbs) $=$ (a \cc{$,$} b) \cc{$::$} (\cf{zip} vas vbs)

\end{block}

Esta la posibilidad de utilizar la funci�n zip para introducir los dotted pattern.

\end{frame}

\begin{frame}
\frametitle{Introducci�n a Agda}
\framesubtitle{Funciones y valores (o pruebas)}


\begin{block}{$\#$-filter}

\cf{$\#$-filter} : \{A : \ct{Set}\} \{n : $\N$\} $\ra$ (A $\ra$ \ct{Bool}) $\ra$ \ct{Vec} A n $\ra$ $\N$\\
\cf{$\#$-filter} p \cc{$[]$} $=$ \cc{zero}\\
\cf{$\#$-filter} p (x \cc{$::$} xs) \ck{with} p x\\
$\ldots$ $|$ \cc{true}  $=$ \cc{suc} (\cf{$\#$-filter} p xs)\\
$\ldots$ $|$ \cc{false} $=$ \cf{$\#$-filter} p xs

\end{block}

\pause

\begin{block}{Filter}

\cf{filter} : \{A : \ct{Set}\} \{n : $\N$\} $\ra$\\
\ \ \ \ \ \ \ \ \ (p : A $\ra$ \ct{Bool}) $\ra$ (xs : \ct{Vec} A n) $\ra$ \ct{Vec} A (\cf{$\#$-filter} p xs)\\
\cf{filter} p \cc{$[]$} $=$ \cc{$[]$}\\
\cf{filter} p (x \cc{$::$} xs) \ck{with} p x\\
$\ldots$ $|$ \cc{true}  $=$ x \cc{$::$} (\cf{filter} p xs)\\
$\ldots$ $|$ \cc{false} $=$ \cf{filter} p xs

\end{block}

\end{frame}

\begin{frame}
\frametitle{Introducci�n a Agda}
\framesubtitle{Funciones y valores (o pruebas)}

\begin{block}{Pares dependientes}\small

Constructor del tipo: Dados tipos A : \ct{Set}, B : A $\ra$ \ct{Set} luego \\
\T \T \T \T \ct{$\Sigma$[} x \ct{$\in$} A \ct{]} B x\\
Constructor de valores: \cc{\_$,$\_}\\
Eliminadores: \\
\T - \cf{proj$_1$} : A\\
\T - \cf{proj$_2$} : B \cf{proj$_1$}

\end{block}

\begin{block}{Filter mejorado}

\cf{filterImproved} : \{A : \ct{Set}\} \{n : $\N$\} $\ra$\\
\T \T \T \T \ \ (A $\ra$ \ct{Bool}) $\ra$ (\ct{Vec} A n) $\ra$ \ct{$\exists$} ($\lambda$ m $\ra$ \ct{Vec} A m)\\
\cf{filterImproved} p \cc{$[]$} $=$ (\cc{zero} \cc{$,$} \cc{$[]$})\\
\cf{filterImproved} p (x \cc{$::$} xs) \ck{with} p x\\
$\ldots$ $|$ \cc{false} $=$ \cf{filterImproved} p xs\\
$\ldots$ $|$ \cc{true} $=$ \ck{let} (n' \cc{$,$} xs') $=$ \cf{filterImproved} p xs\\
\T \T \T  \ \ck{in} (\cc{suc} n' \cc{$,$} x \cc{$::$} xs')
\end{block}

\end{frame}

\begin{frame}
\frametitle{Introducci�n a Agda}
\framesubtitle{Funciones y valores (o pruebas)}

\begin{block}{Naturales}

\ck{data} \ct{\_$\leq$\_} : $\N$ $\ra$ $\N$ $\ra$ \ct{Set} \ck{where}\\
\ \ \ \ \cc{z$\leq$n} : \{n : $\N$\}                 $\ra$ \cc{zero}  \ct{$\leq$} n\\
\ \ \ \ \cc{s$\leq$s} : \{m n : $\N$\} $\ra$ (m$\leq$n : m \ct{$\leq$} n) $\ra$ \cc{suc} m \ct{$\leq$} \cc{suc} n

\end{block}

\pause

\begin{block}{Propiedad sobre el orden de los naturales}

\cf{prop$\leq$} : \{m n : $\N$\} $\ra$ m \ct{$\leq$} n $\ra$ m \cf{prop$\leq$} \cc{suc} n\\
\cf{prop$\leq$} \cc{z$\leq$n} = \cc{z$\leq$n}\\
\cf{prop$\leq$} (\cc{s$\leq$s} m$\leq$n) = \cc{s$\leq$s} (\cf{prop$\leq$} m$\leq$n)

\end{block}

\end{frame}

\begin{frame}
\frametitle{Introducci�n a Agda}
\framesubtitle{Funciones y valores (o pruebas)}

\begin{block}{filter mejorado y con prueba}

\cf{filterImproved'} : \{A : \ct{Set}\} \{n : $\N$\} $\ra$\\
\T \T \T \T \ \ 
(A $\ra$ \ct{Bool}) $\ra$ (\ct{Vec} A n) $\ra$\\
\T \T \T \T \ \ 
\ct{$\exists$} ($\lambda$ m $\ra$ \ct{Vec} A m \ct{$\times$} m \ct{$\leq$} n)\\
\cf{filterImproved'} p \cc{$[]$} $=$ (\cc{zero} \cc{$,$} \cc{$[]$} \cc{$,$} \cc{z$\leq$n})\\
\cf{filterImproved'} p (x \cc{$::$} xs) \ck{with} p x $|$ \cf{filterImproved'} p xs\\
$\ldots$ $|$ \cc{false} $|$ (n' \cc{$,$} xs' \cc{$,$} m$\leq$n) = 
(n' \cc{$,$} xs' \cc{$,$} \cf{prop$\leq$} m$\leq$n)\\
$\ldots$ $|$ \cc{true}  $|$ (n' \cc{$,$} xs' \cc{$,$} m$\leq$n) = 
(\cc{suc} n' , x \cc{$::$} xs' , \cc{s$\leq$s} m$\leq$n)

\end{block}

\end{frame}

\end{document}

